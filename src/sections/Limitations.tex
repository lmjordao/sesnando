\label{ch:2}
\chapter{Limitations of this study}

This software application will be firstly focused on a mature railway project where already advanced development has been made. Its requirements will be used as a mold for the creation of the predefined grammar of the acceptable requirements by the application under development (SESNANDO). It is intended to develop a core grammar that can interpret requirements from multiple railway projects. However, at the current state of the development made, using SESNANDO for a different project might require a slightly modified grammar to be created and installed. \\

By the time the developments of this software application started, there were already a significant number of requirements written on the company's client side involving natural language that were not precisely aligned with the internal defined guidelines. For the verification of this tool, such requirements needed to be re-written in a format that SESNANDO could interpret them.\\

It is intended (as discussed internally) that this software could be used on a railway project for Infraestruturas de Portugal. By the time of the writing of this report, this project is still in early development days, thus, for the validation of this tool, and as the scope of this thesis the former railway project will be used.\\
