\label{ch:3}
\chapter{State of the art}

STATE OF THE ART\\
o que existe no mercado\\
porque é que não serve\\
o que se pretende optimizar\\
porque é que o sesnando vai ser melhor.

%----------------------- Software requirements ----------------------------
\section{Software Requirements}
\label{sec:software_requirements}

Software Requirements are a set of statements contained on a Software Requirements Specification (SRS) that describe a system behaviour or functionality. Usually, software requirements are a low-level design of the business requirements. The process of translate, document and write such requirements are defined as Requirements Engineering. FONTE\\
Requirements that perfectly define a system are hard to write, as such, some principles have been designed, e.g. the \textit{INCOSE Guide for Writing Requirements} or the \textit{S.M.A.R.T} criteria in order to improve how they are written, so they are strongly understandable by a human or a machine.\\
Requirements are written prior to the development phase and are usually written by the Developers and Requirement Managers. Errors introduced during the Requirements Engineering phase have exponential costs on later phases, thus, writing precise requirements is important. \\
Error-free requirements are the foundations on which the code should be build and which the code should be tested. If the requirements are faulty, the code is likely to be faulty and tests of the code will be impossible, will fail or give meaningless results. The effort to produce error-free requirements is considerable, but is nevertheless smaller that the effort to answer questions and to correct the requirements, the code and its tests because the original requirements were faulty.


\label{sec:requirement_specification}
\section{Requirement Specification Techniques}

Behaviour Driven Development\\
Test-Driven Development (foco no facto do desenvolvimento ser orientado aos testes e não aos requisitos)\\
Describe the Gherkin language.\\
Sesnando uses the gherkin language structure.\\
The gherkin language defines the main requirement skeleton\\
Given são pre-condições, When representa uma mudança de estado ou trigger, Then representa como o sistema ou software se deve comportar\\


%----------------------- Software testing ----------------------------
\section{Software testing}
\label{sec:software_testing}

The purpose of software testing activities is to verify that a system is operating according to predefined design and meets the customer needs.
Testing can occur on different levels 
\\
\textcolor{red}{
Notes:\\
what is the purpose of testing activities\\
who should be testing\\
who writes tests\\
why are tests so important ... (without tests the software would be prone to errors)\\
FALAR DO V-MODEL\\
introduzir paths and branchs
}


\label{sec:black-box testing}
\section{Black-box testing}

Cause-Effect graph 

\label{sec:requirement_testing}
\section{Requirement Testing Tools}

Cucumber\\
Cucumber is a tool that supports behaviour-driven development. It reads Test definitions and test steps written using the Gherkin language \cite{cucumber}.\\

Jest\\
Cypress\\


\label{sec:computational_linguistics}
\section{Computational Linguistics and Language Recognition}


Chomsky types\\
chomsky types 0 to type 3\\
LTR and LL(*)\\
bottom-up and top-down\\
lexers\\
parsers\\
Falar do ANTLR\\
Explicar o que é uma AST\\
Explicar o que é o antlr\\
porquê o antlr\\



\section{Related Works}
\label{sec:related_works}

Artigos\\


\section{Conclusions on state of the art}
\label{sec:sota_conclusions}

Alguns artigos lidos tentam chegar à interpretação de linguagem natural, que pode induzir em erro, ou definem uma linguagem natural controlada mas dependem de muitos inputs do utilizador, outros artigos pedem ao utilizador para atribuir significado às palavras na altura do processamento do requisito.

