\chapter*{Abstract}
\label{ch:abstract}

Software and system testing is an intrinsic activity in software development. It is estimated that a great effort in the development of a system lies in the verification and validation of it. The scope of this report is the development of a tool capable of generating sets of test cases by interpreting software requirements written in a predefined format, thus allowing to greatly reduce the costs associated with the testing activities.\\
Sesnando is a tool developed at Critical Software, S.A. whose objective is to interpret and compile requirements written in a controlled natural language and from these, automatically generate a set of tests or test specification that allow verifying the realisation or implementation of this same requirement. During the interpretation phase of the given requirement, \textit{Sesnando} acts as a validator for writing the requirement and provides messages to the user about its construction. On a successful compilation, \textit{Sesnando} connects to a remote service to obtain additional information about the requirements. The test specification generated by \textit{Sesnando}, that contains the definition of the values of the signals to be assigned in the system as well as the expected results (observation of the behavior of the system), also in the form of signals (signal read) is composed of test steps. Each test step writes and reads a set of system signals. The set of these test steps is generated according to the test coverage criteria that must be applied, in this case the MCDC. The results obtained show that it is possible to reduce the effort in the system testing activity by up to 90\% per requirement.




