\chapter*{Resumo}
\label{ch:summary}

O Sesnando é uma ferramenta desenvolvida na Critical Software, S.A. cujo objectivo é interpretar e compilar requisitos escritos numa linguagem natural controlada e a partir destes gerar automaticamente um conjunto de testes ou especificação de testes que permitam verificar a realisação ou implementação deste mesmo requisito. Durante a fase de iterpretação do requisito dado, o Sesnando age como um validador da escrita do requisito e fornece mensagens ao utilizador sobre a sua construção. Numa compilação bem sucedida, o Sesnando liga-se a um serviço remoto para obter informação adicional sobre os requisitos, nomeadamente, em função dos valores das condições dos requisitos que variáveis do sistema forçar para chegar a sua verificação. A especificação de testes gerada pelo Sesnando que contêm a definição dos valores dos sinais a atribuir no sistema assim como os resultados esperados (obseração do comportamento do sistema), também estes em forma de sinais (leitura) é composta por passos de teste. Cada passo de teste escreve e lê um conjunto de sinais do sistema. O conjunto destes passos de teste é gerado em função do critério de teste que deve ser aplicado, neste caso o MCDC. Os resultados obtidos mostram que é possivel reduzir o esforço na actividade de testes de sistema em até 90\% por requisito.