\chapter*{Conclusion}
\label{ch:conclusion}

\textit{Sesnando} has been presented at Critical Software and received positive feedback. However, we believe that there is still a lot of room for improvement.
Such improvements are towards the interpretation and validation of all kinds of requirement conditions such as time-frames and requirements that depend on each other (e.g. the output of one or more requirements are the inputs of another), in other words, all requirements that are written under the \textit{GIVEN WHEN THEN} blueprint, not only on the Railway markets, but on all ASDT (Aerospace Defense and transportation) markets.
Moreover, it is ambitioned that \textit{Sesnando} could be used on projects that do not solely rely on a signal based architecture system, for instance, functional testing for Desktop and Web Applications. Given this realisation, \textit{Sesnando} could be a great tool that supports any tester in its daily activities. It is believed that there is a place out there for \textit{Sesnando} in the world of behavior-driven-development, given that once a requirement is written a test specification can be instantly generated, and used to validate such requirement against the existing software. Therefore, the following future work is proposed. 

\section*{Future work}
\label{sec:future_work}

\begin{itemize}
\item Development of a user interface or a plugin for a text editor able to point out syntax errors and warnings on requirements.
\item Improvement of the user interface to display test specifications and the ability to export test scripts in multiple formats.
\item Checks for conflicting requirements, i.e. requirements that contradict each other, facilitating the process of requirement review.
\item Ability to include more than one requirement into a single specification, by detecting requirements that depend on each other.
\item Ability to define Features and Scenarios in order to comply with the Gerkin syntax as Gerkin also implements the \textit{GIVEN WHEN THEN} blueprint, \textit{Sesnando} could, for instance be used as a Cucumber engine for Signal based systems. 
\item Ability to pull and push the remote signal database, so \textit{Sesnando} could be used offline.
\item Connection with Jira, so the requirements could be exported automatically and having their specifications uploaded. 
\end{itemize}