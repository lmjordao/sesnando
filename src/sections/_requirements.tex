%----------------------- Requirement analysis ----------------------------
\section{Requirement analysis}
\label{sec:requirement_analysis}

The activity of identifying software functionalities and behavior is described as \textit{Requirements Engineering}. Before proceeding to the presentation of the \textit{Sesnando} architecture, the requirements of the \textit{Sesnando} tool (denominated as 'The application') will be identified in a simple manner.

\begin{itemize}
\item The application must be able to recognize a requirement identified by its keyword REQUIREMENT.
\item The application must be able to identify \textit{Given}, \textit{When} and \textit{Then} predicates.
\item The application must extract Logical Expressions between \textit{AND} and \textit{OR} boolean operators.
\item The application must identify Logical Expressions using the operators: \textit{Equal than}, \textit{Lower Than}, \textit{Lower Than or Equal}, \textit{Greater Than} and \textit{Greater Than or Equal}.
\item The application must be able to identify Requirement Signals (Logical Signals) within a Logical Expression.
\item The application must be able to identify an Operand within a Logical Expression.
\item The application must be able to identify a Logical Expression Quantifier, i.e., the system components to which the expression shall be evaluated, e.g., a train axle.
\item The application must be able to compile the requirement into a Parse tree.
\item The application must be able to access the Parse Tree and generate a set of test cases from the interpreted requirements.
\item The application must be able to generate a standard output file (CSV) containing a set of test cases.
\item The application must be able to generate a test script to be used on the testing environment.
\end{itemize}